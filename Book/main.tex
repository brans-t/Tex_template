% ======================== main.tex ========================
% 主控文件:整本书的入口
% ----------------------------------------------------------
% 定义 \allfiles 控制分章节编译
\def\allfiles{}

% ----------------------------------------------------------
% 文档类
\documentclass[12pt, a4paper, oneside, UTF8]{ctexbook}

% 全局路径变量(方便修改)
\def\path{./config}

% 导入全局配置(字体、宏包、环境等)
% ====================== _config.tex ======================
% 文档配置与全局参数定义
% ----------------------------------------------------------

% -------------------- 文档元信息 --------------------
\def\myTitle{主标题}               % 文档标题
\def\myAuthor{author}                                   % 作者
\def\myDateCover{封面日期:\today}                     % 封面日期
\def\myDateForeword{前言页显示日期:\today}             % 前言日期
\def\mySubheading{——副标题}               % 副标题

% -------------------- 前言内容 --------------------
\def\myForeword{前言标题}                              % 前言标题
\def\myForewordText{                                   % 前言正文
    这里是前言内容。
}

% -------------------- 导入全局宏包 --------------------
% package.tex

% =============================
% 数学相关宏包
% =============================
\usepackage{amsmath}          % 提供各种数学环境和命令
\usepackage{amssymb}          % 数学符号
\usepackage{amsthm}           % 定理、引理等环境
\usepackage{bm}               % 数学粗体
\usepackage{mathrsfs}         % 花体字母

% =============================
% 页面布局与格式
% =============================
\usepackage{geometry}         % 页面布局
\usepackage{setspace}         % 行距控制
\usepackage{graphicx}         % 插入图片
\usepackage{caption}          % 图表标题格式
\usepackage{subcaption}       % 子图标题
\usepackage{placeins}         % 控制浮动体位置
\usepackage{booktabs}         % 高质量表格线
\usepackage{enumitem}         % 自定义列表样式
\usepackage[bottom]{footmisc} % 脚注靠底
\usepackage{fancyhdr}         % 页眉页脚

% =============================
% 颜色与样式
% =============================
\usepackage[dvipsnames, svgnames]{xcolor} % 定义颜色
\usepackage{pifont}           % 特殊符号
\usepackage{tcolorbox}        % 彩色框
\tcbuselibrary{skins,breakable}

% =============================
% 代码与算法
% =============================
\usepackage{listings}         % 代码高亮
\usepackage{algorithm}        % 算法环境
\usepackage[noend]{algpseudocode} % 伪代码,无 end 标记
\usepackage{etoolbox}         % 宏编程工具

% =============================
% 其他功能
% =============================
\usepackage{hyperref}         % 超链接与交叉引用
\usepackage{url}              % 格式化 URL 和 DOI
\usepackage{float}            % 浮动体控制
\usepackage{appendix}         % 附录
\usepackage{animate}          % 动画支持


% =============================
% 可选功能(按需启用)
% =============================
\usepackage{tikz}           % 绘图
% \usepackage[most]{tcolorbox} % 更多 tcolorbox 功能
                              % 宏包加载

% -------------------- 导入环境配置 --------------------
% ======================
% 定理与盒子环境设置
% 依赖宏包:
%   amsthm, amssymb       (定理、证明环境)
%   xcolor                (颜色定义)
%   tcolorbox[skins,breakable] (彩色盒子)
% ======================

% ---------- 自定义颜色 ----------
\definecolor{greenshade}{rgb}{0.9, 1, 0.9}
\definecolor{green}{rgb}{0.2, 0.6, 0.2}
\definecolor{axmBack}{rgb}{0.95, 0.95, 1}
\definecolor{axmFrame}{rgb}{0.5, 0.5, 1}
\definecolor{axmTitle}{rgb}{0.2, 0.2, 0.6}
\definecolor{lightblue}{rgb}{0.8, 0.9, 1}
\definecolor{lightgray}{rgb}{0.95, 0.95, 0.95}
\definecolor{lightyellow}{rgb}{1, 0.98, 0.9}
\definecolor{lightcyan}{rgb}{0.9, 1, 1}
\definecolor{SkyBlue}{rgb}{0.2,0.6,1}

% ---------- amsthm 样式 ----------
% 斜体样式(定理、引理、命题、推论)
\newtheoremstyle{mythm}
  {\topsep}{\topsep}{\itshape}{}{\bfseries}{.}{.5em}{}
\theoremstyle{mythm}
\newtheorem{theorem}{定理}[chapter]
\newtheorem{lemma}[theorem]{引理}
\newtheorem{corollary}[theorem]{推论}
\newtheorem{proposition}[theorem]{命题}
\newtheorem{axiom}{公理}[chapter]        % 新增 axiom 环境

% 非斜体样式(定义、例、注)
\newtheoremstyle{mydef}
  {\topsep}{\topsep}{\upshape}{}{\bfseries}{.}{.5em}{}
\theoremstyle{mydef}
\newtheorem{definition}{定义}[chapter]
\newtheorem{example}{例}[chapter]
\newtheorem{remark}{注}[chapter]

% ---------- 自定义证明与解答环境 ----------
\renewenvironment{proof}{\indent\textcolor{SkyBlue}{\textbf{证明.}}\;}{\qed\par}
\newenvironment{solution}{\indent\textcolor{SkyBlue}{\textbf{解.}}\;}{\qed\par}

% ---------- tcolorbox 自定义环境 ----------
% 定义彩色盒子环境
\newtcolorbox[auto counter, number within=chapter, list inside=mydefbox]{defn}[1][]{
  enhanced, breakable,
  colback=greenshade, colframe=green, arc=3mm,
  title={\textbf{定义\thetcbcounter\ #1}},
  coltitle=black, fonttitle=\bfseries
}

% axiom 盒子
\newtcolorbox[auto counter, number within=chapter]{axiombox}[1][]{
  enhanced, breakable,
  colback=axmBack, colframe=axmFrame, arc=3mm,
  title={\textbf{公理\thetcbcounter\ #1}},
  coltitle=axmTitle, fonttitle=\bfseries,
}

% 注意框
\newtcolorbox{Notice}[1][注意]{
  colback=lightblue, colframe=black, arc=3mm,
  title={\textbf{#1}}, coltitle=white, fonttitle=\bfseries
}

% 证明框
\newtcolorbox{Proof}{
  colback=lightgray, colframe=black, arc=3mm,
  title={\textbf{证明}}, coltitle=black, fonttitle=\bfseries
}

% 推导框
\newtcolorbox{Derivation}[1][]{
  enhanced, breakable,
  colback=lightyellow, colframe=black, arc=3mm,
  title={\textbf{推导}}, coltitle=white, fonttitle=\bfseries,
  boxsep=2mm, #1
}

% 示例框
\newtcolorbox{examplebox}[1][示例]{
  colback=lightcyan, colframe=black, arc=3mm,
  title={\textbf{#1}}, coltitle=black, fonttitle=\bfseries
}

% ---------- 快捷别名 ----------
% \let\defn\definition
% \newenvironment{axiom}[1][]{\begin{axiombox}[#1]}{\end{axiombox}}
                 % 定理环境
% ======================
% 全局代码样式(基于 listings)
% ======================

\definecolor{codebg}{RGB}{245,245,245}       % 代码背景色
\definecolor{codeframe}{RGB}{220,220,220}    % 代码边框色

\lstset{
  basicstyle=\small\ttfamily,                         % 基本字体样式:小号等宽字体
  backgroundcolor=\color{codebg},                    % 代码背景色
  frame=single,                                      % 代码框架类型:单线框
  framesep=5pt,                                      % 代码框与框线的间距
  rulecolor=\color{codeframe},                       % 框线颜色
  framerule=0.8pt,                                   % 框线宽度
  %
  numbers=left,                                      % 行号显示在左边
  numberstyle=\footnotesize\color{gray},                     % 行号样式:灰色小号字体
  numbersep=5pt,                                     % 行号与代码的间距
  stepnumber=1,                                      % 每隔几行显示一个行号
  firstnumber=1,                                     % 起始行号
  %
  tabsize=4,                                         % Tab 等于几个空格
  showspaces=false,                                  % 不显示空格符号
  showstringspaces=false,                            % 字符串中的空格不显示特殊符号
  showtabs=false,                                    % 不显示 Tab 符号
  breaklines=true,                                   % 自动换行
  breakatwhitespace=false,                           % 是否仅在空格处换行
  prebreak=\raisebox{0ex}[0ex][0ex]{\ensuremath{\hookleftarrow}}, % 换行前标记符
  %
  keywordstyle=\color[RGB]{0,0,255}\bfseries,       % 关键字样式:蓝色加粗
  commentstyle=\color[RGB]{34,139,34}\itshape,      % 注释样式:绿色斜体
  stringstyle=\color[RGB]{178,34,34},               % 字符串样式:红色
  identifierstyle=\color{black},                    % 标识符样式:黑色
  %
  captionpos=b,                                      % 标题位置:下方
  xleftmargin=15pt,                                  % 左边距
  xrightmargin=5pt,                                  % 右边距
  keepspaces=true,                                   % 保留空格
  % escapeinside={@}{@},                               % 在代码中可使用 @…@ 转义 LaTeX
  aboveskip=1em,                                     % 代码块上方间距
  belowskip=1em                                      % 代码块下方间距(注意最后一行不要逗号)
}


% ======================
% 语言定义
% ======================

% --- Python ---
\lstdefinelanguage{Python}{
  keywords={False,None,True,and,as,assert,async,await,break,class,continue,def,
            del,elif,else,except,finally,for,from,global,if,import,in,is,lambda,
            nonlocal,not,or,pass,raise,return,try,while,with,yield},
  keywords=[2]{self,cls,__init__,__name__,__main__},
  keywords=[3]{print,input,len,str,int,float,list,dict,set,tuple,range},
  sensitive=true,
  comment=[l]{\#},
  morecomment=[s]{"""}{"""},
  morecomment=[s]{'''}{'''},
  string=[b]',
  string=[b]",
  morestring=[s]{r'}{'},
  morestring=[s]{r"}{"}
}

% --- MATLAB ---
\lstdefinelanguage{Matlab}{
  keywords={break,case,catch,classdef,continue,else,elseif,end,for,function,
            global,if,otherwise,parfor,persistent,return,spmd,switch,try,while},
  keywords=[2]{pi,eps,inf,nan,ans},
  keywords=[3]{plot,sin,cos,exp,log,mean,std},
  sensitive=false,
  comment=[l]{\%},
  morecomment=[s]{\%\{}{\%\}},
  string=[b]',
  string=[b]"
}

% --- C ---
\lstdefinelanguage{C}{
  keywords={auto,break,case,char,const,continue,default,do,double,else,enum,
            extern,float,for,goto,if,inline,int,long,register,restrict,return,
            short,signed,sizeof,static,struct,switch,typedef,union,unsigned,
            void,volatile,while,_Bool,_Complex,_Imaginary},
  sensitive=true,
  morecomment=[l]{//},
  morecomment=[s]{/*}{*/},
  string=[b]',
  string=[b]",
  alsoletter=\#,
  morekeywords=[2]{printf,scanf,malloc,free,sizeof}
}

% --- Mathematica ---
\lstdefinelanguage{Mathematica}{
  keywords={If,Do,For,While,Module,Block,With,Catch,Throw,Return,Break,Continue,
            Function,Pattern,Set,SetDelayed,Rule,RuleDelayed,ReplaceAll,
            ReplaceRepeated,True,False,Null,List,Table,Map,Apply,Flatten},
  morekeywords=[2]{Sin,Cos,Log,Integrate,D,Plot,Solve,Expand},
  sensitive=true,
  comment=[l]{(*)},
  morecomment=[l]{*)},
  string=[b]",
  alsoletter={@,?,\#},
  morestring=[b]`
}

% --- R ---
\lstdefinelanguage{R}{
  keywords={function,if,in,else,for,while,repeat,break,next,return,switch,
            try,tryCatch,stop,warning,require,library,attach,detach,source},
  keywords=[2]{c,matrix,data.frame,list,vector,array,table},
  keywords=[3]{mean,sd,plot,hist,lm,glm},
  sensitive=true,
  comment=[l]{\#},
  string=[b]',
  string=[b]",
  otherkeywords={!,!=,~,$,*,\&,\%/\%,\%*\%,\%\%,<-,<<-,->,->>,=,==,>,<,>=,<=}
}

% ======================
% 可分割算法环境
% ======================
\makeatletter
\newenvironment{breakablealgorithm}
  {% \begin{breakablealgorithm}
    \begin{center}
      \refstepcounter{algorithm}% New algorithm
      \hrule height.8pt depth0pt \kern4pt% \@fs@pre for \@fs@ruled
      \parskip 0pt
      \renewcommand{\caption}[2][\relax]{% Make a new \caption
        {\raggedright\textbf{\fname@algorithm~\thealgorithm: ##2}\par}%
        \ifx\relax##1\relax % #1 is \relax
          \addcontentsline{loa}{algorithm}{\protect\numberline{\thealgorithm}##2}%
        \else % #1 is not \relax
          \addcontentsline{loa}{algorithm}{\protect\numberline{\thealgorithm}##1}%
        \fi
        \kern2pt\hrule\kern2pt
     }
  }
  {% \end{breakablealgorithm}
     \kern2pt\hrule\relax% \@fs@post for \@fs@ruled
   \end{center}
  }
\makeatother

% 算法列表
\providecommand{\listofalgorithms}{\listof{algorithm}{List of Algorithms}}

% ======================
% 自定义代码环境
% ======================
\lstnewenvironment{pythoncode}[1][]    {\lstset{language=Python,#1}}{}
\lstnewenvironment{matlabcode}[1][]    {\lstset{language=Matlab,#1}}{}
\lstnewenvironment{ccode}[1][]         {\lstset{language=C,#1}}{}
\lstnewenvironment{mathematicacode}[1][]{\lstset{language=Mathematica,#1}}{}
\lstnewenvironment{rcode}[1][]         {\lstset{language=R,#1}}{}

% ======================
% 辅助命令
% ======================
\newcommand{\inlinecode}[1]{\colorbox{codebg}{\ttfamily #1}}
\newcommand{\inputcode}[2][]{\lstinputlisting[language=#1]{#2}}
                    % 代码环境
% % ======================
% 图表与浮动体设置
% ======================

% ---------- 标题样式 ----------
% 依赖 package.tex 中的 \usepackage{caption}
\captionsetup[figure]{
  font={small},          % 标题字体
  labelfont={bf},        % 标签加粗
  justification=centering % 居中对齐
}
\captionsetup[table]{
  font={small},
  labelfont={bf},
  justification=centering
}

% ---------- 自定义浮动体 ----------
% 依赖 package.tex 中的 \usepackage{float}
\newfloat{code}{htbp}{loc} % 定义 "code" 浮动体
\floatname{code}{程序}     % 浮动体标题名

% ---------- 自定义图片环境 ----------
\newenvironment{myfigure}[1][]{
  \begin{figure}[#1]
  \centering
}{
  \end{figure}
}
                  % 图片环境
% ======================
%  自定义样式与环境
% ======================

% ---------- 颜色定义 ----------
\usepackage{xcolor}
\definecolor{highlightcolor}{RGB}{255,230,179} % 淡黄高亮
\definecolor{myred}{RGB}{204,26,26}           % 强调文字

% ---------- 文字样式 ----------
\usepackage[normalem]{ulem}          % 波浪线 \uwave
\newcommand{\hl}[1]{\colorbox{highlightcolor}{#1}}
% \newcommand{\uline}[1]{\underline{#1}}
\newcommand{\wline}[1]{\uwave{#1}}
\newcommand{\myemph}[1]{\textcolor{myred}{#1}}

% ---------- 超链接 ----------
\usepackage{hyperref}
\hypersetup{
    colorlinks   = true,
    linkcolor    = black,
    citecolor    = black,
    urlcolor     = blue,
    filecolor    = magenta,
    pdfstartview = Fit,
    breaklinks   = true
}

% ---------- 行距 ----------
\usepackage{setspace}
\onehalfspacing                    % 1.5 倍行距
\setlength{\headheight}{14.5pt}   % fancyhdr 推荐值

% ---------- 图片搜索路径 ----------
\usepackage{graphicx}
\graphicspath{%
  {config/cover/}%
  {Chapters/chapter1/images/}%
  {Chapters/chapter2/images/}%
  {Chapters/chapter3/images/}%
  {images/}%
}
                  % 自定义文字环境

% -------------------- 封面模板选择 --------------------
% 默认使用索引为 1 的封面配置
\def\myIndex{1}
\input{\path/cover/cover_package_\myIndex.tex}

% ====================== End of _config.tex ======================


% ----------------------------------------------------------
\begin{document}

% 封面
% ====================== cover.tex ======================
% 封面与前言组合文件
% -------------------------------------------------------
% 1. 载入封面文字内容(标题、作者、图片等)
\input{\path/cover/cover_text_\myIndex.tex}

% -------------------------------------------------------
% 2. 空白页:版权声明 / 空页 / 致谢等
\newpage
\thispagestyle{empty}       % 不显示页眉页脚
\begin{center}
    \Huge\bfseries \myForeword  % 居中显示“前言”/“序”等标题
\end{center}
\myForewordText              % 正文:由外部宏 \myForewordText 提供
\begin{flushright}
    \begin{tabular}{c}
        \myDateForeword     % 作者署名与日期
    \end{tabular}
\end{flushright}

% -------------------------------------------------------
% 3. 目录页
\newpage
\pagestyle{plain}           % 普通页眉页脚(仅页码)
\setcounter{page}{1}        % 重新计数
\pagenumbering{Roman}       % 目录页码使用大写罗马数字 I, II, III …
% \tableofcontents            % 生成目录

% -------------------------------------------------------
% 4. 正文起始页
\newpage
\pagenumbering{arabic}      % 正文页码使用阿拉伯数字 1, 2, 3 …
\setcounter{page}{1}        % 从第 1 页开始
\setcounter{chapter}{-1}    % 第零章(可选)

% 载入 fancy 页眉页脚样式
\pagestyle{fancy}
\fancyfoot[C]{\thepage}     % 页脚居中显示页码
\renewcommand{\headrulewidth}{0.4pt} % 页眉横线粗细
\renewcommand{\footrulewidth}{0pt}   % 页脚横线粗细(0pt 表示无横线)

% 目录
\tableofcontents
\clearpage

% ======================== 正文部分 ========================
% ====================== chapter1/chap1.tex ======================
% 第一章:引言
% 支持单独编译与整书编译两种模式
% ---------------------------------------------------------------

\ifx\allfiles\undefined
    % ======= 单独编译模式 =======
    \documentclass{ctexbook}

    % 载入全局配置(路径需根据文件结构调整)
    % ====================== _config.tex ======================
% 文档配置与全局参数定义
% ----------------------------------------------------------

% -------------------- 文档元信息 --------------------
\def\myTitle{主标题}               % 文档标题
\def\myAuthor{author}                                   % 作者
\def\myDateCover{封面日期:\today}                     % 封面日期
\def\myDateForeword{前言页显示日期:\today}             % 前言日期
\def\mySubheading{——副标题}               % 副标题

% -------------------- 前言内容 --------------------
\def\myForeword{前言标题}                              % 前言标题
\def\myForewordText{                                   % 前言正文
    这里是前言内容。
}

% -------------------- 导入全局宏包 --------------------
% package.tex

% =============================
% 数学相关宏包
% =============================
\usepackage{amsmath}          % 提供各种数学环境和命令
\usepackage{amssymb}          % 数学符号
\usepackage{amsthm}           % 定理、引理等环境
\usepackage{bm}               % 数学粗体
\usepackage{mathrsfs}         % 花体字母

% =============================
% 页面布局与格式
% =============================
\usepackage{geometry}         % 页面布局
\usepackage{setspace}         % 行距控制
\usepackage{graphicx}         % 插入图片
\usepackage{caption}          % 图表标题格式
\usepackage{subcaption}       % 子图标题
\usepackage{placeins}         % 控制浮动体位置
\usepackage{booktabs}         % 高质量表格线
\usepackage{enumitem}         % 自定义列表样式
\usepackage[bottom]{footmisc} % 脚注靠底
\usepackage{fancyhdr}         % 页眉页脚

% =============================
% 颜色与样式
% =============================
\usepackage[dvipsnames, svgnames]{xcolor} % 定义颜色
\usepackage{pifont}           % 特殊符号
\usepackage{tcolorbox}        % 彩色框
\tcbuselibrary{skins,breakable}

% =============================
% 代码与算法
% =============================
\usepackage{listings}         % 代码高亮
\usepackage{algorithm}        % 算法环境
\usepackage[noend]{algpseudocode} % 伪代码,无 end 标记
\usepackage{etoolbox}         % 宏编程工具

% =============================
% 其他功能
% =============================
\usepackage{hyperref}         % 超链接与交叉引用
\usepackage{url}              % 格式化 URL 和 DOI
\usepackage{float}            % 浮动体控制
\usepackage{appendix}         % 附录
\usepackage{animate}          % 动画支持


% =============================
% 可选功能(按需启用)
% =============================
\usepackage{tikz}           % 绘图
% \usepackage[most]{tcolorbox} % 更多 tcolorbox 功能
                              % 宏包加载

% -------------------- 导入环境配置 --------------------
% ======================
% 定理与盒子环境设置
% 依赖宏包:
%   amsthm, amssymb       (定理、证明环境)
%   xcolor                (颜色定义)
%   tcolorbox[skins,breakable] (彩色盒子)
% ======================

% ---------- 自定义颜色 ----------
\definecolor{greenshade}{rgb}{0.9, 1, 0.9}
\definecolor{green}{rgb}{0.2, 0.6, 0.2}
\definecolor{axmBack}{rgb}{0.95, 0.95, 1}
\definecolor{axmFrame}{rgb}{0.5, 0.5, 1}
\definecolor{axmTitle}{rgb}{0.2, 0.2, 0.6}
\definecolor{lightblue}{rgb}{0.8, 0.9, 1}
\definecolor{lightgray}{rgb}{0.95, 0.95, 0.95}
\definecolor{lightyellow}{rgb}{1, 0.98, 0.9}
\definecolor{lightcyan}{rgb}{0.9, 1, 1}
\definecolor{SkyBlue}{rgb}{0.2,0.6,1}

% ---------- amsthm 样式 ----------
% 斜体样式(定理、引理、命题、推论)
\newtheoremstyle{mythm}
  {\topsep}{\topsep}{\itshape}{}{\bfseries}{.}{.5em}{}
\theoremstyle{mythm}
\newtheorem{theorem}{定理}[chapter]
\newtheorem{lemma}[theorem]{引理}
\newtheorem{corollary}[theorem]{推论}
\newtheorem{proposition}[theorem]{命题}
\newtheorem{axiom}{公理}[chapter]        % 新增 axiom 环境

% 非斜体样式(定义、例、注)
\newtheoremstyle{mydef}
  {\topsep}{\topsep}{\upshape}{}{\bfseries}{.}{.5em}{}
\theoremstyle{mydef}
\newtheorem{definition}{定义}[chapter]
\newtheorem{example}{例}[chapter]
\newtheorem{remark}{注}[chapter]

% ---------- 自定义证明与解答环境 ----------
\renewenvironment{proof}{\indent\textcolor{SkyBlue}{\textbf{证明.}}\;}{\qed\par}
\newenvironment{solution}{\indent\textcolor{SkyBlue}{\textbf{解.}}\;}{\qed\par}

% ---------- tcolorbox 自定义环境 ----------
% 定义彩色盒子环境
\newtcolorbox[auto counter, number within=chapter, list inside=mydefbox]{defn}[1][]{
  enhanced, breakable,
  colback=greenshade, colframe=green, arc=3mm,
  title={\textbf{定义\thetcbcounter\ #1}},
  coltitle=black, fonttitle=\bfseries
}

% axiom 盒子
\newtcolorbox[auto counter, number within=chapter]{axiombox}[1][]{
  enhanced, breakable,
  colback=axmBack, colframe=axmFrame, arc=3mm,
  title={\textbf{公理\thetcbcounter\ #1}},
  coltitle=axmTitle, fonttitle=\bfseries,
}

% 注意框
\newtcolorbox{Notice}[1][注意]{
  colback=lightblue, colframe=black, arc=3mm,
  title={\textbf{#1}}, coltitle=white, fonttitle=\bfseries
}

% 证明框
\newtcolorbox{Proof}{
  colback=lightgray, colframe=black, arc=3mm,
  title={\textbf{证明}}, coltitle=black, fonttitle=\bfseries
}

% 推导框
\newtcolorbox{Derivation}[1][]{
  enhanced, breakable,
  colback=lightyellow, colframe=black, arc=3mm,
  title={\textbf{推导}}, coltitle=white, fonttitle=\bfseries,
  boxsep=2mm, #1
}

% 示例框
\newtcolorbox{examplebox}[1][示例]{
  colback=lightcyan, colframe=black, arc=3mm,
  title={\textbf{#1}}, coltitle=black, fonttitle=\bfseries
}

% ---------- 快捷别名 ----------
% \let\defn\definition
% \newenvironment{axiom}[1][]{\begin{axiombox}[#1]}{\end{axiombox}}
                 % 定理环境
% ======================
% 全局代码样式(基于 listings)
% ======================

\definecolor{codebg}{RGB}{245,245,245}       % 代码背景色
\definecolor{codeframe}{RGB}{220,220,220}    % 代码边框色

\lstset{
  basicstyle=\small\ttfamily,                         % 基本字体样式:小号等宽字体
  backgroundcolor=\color{codebg},                    % 代码背景色
  frame=single,                                      % 代码框架类型:单线框
  framesep=5pt,                                      % 代码框与框线的间距
  rulecolor=\color{codeframe},                       % 框线颜色
  framerule=0.8pt,                                   % 框线宽度
  %
  numbers=left,                                      % 行号显示在左边
  numberstyle=\footnotesize\color{gray},                     % 行号样式:灰色小号字体
  numbersep=5pt,                                     % 行号与代码的间距
  stepnumber=1,                                      % 每隔几行显示一个行号
  firstnumber=1,                                     % 起始行号
  %
  tabsize=4,                                         % Tab 等于几个空格
  showspaces=false,                                  % 不显示空格符号
  showstringspaces=false,                            % 字符串中的空格不显示特殊符号
  showtabs=false,                                    % 不显示 Tab 符号
  breaklines=true,                                   % 自动换行
  breakatwhitespace=false,                           % 是否仅在空格处换行
  prebreak=\raisebox{0ex}[0ex][0ex]{\ensuremath{\hookleftarrow}}, % 换行前标记符
  %
  keywordstyle=\color[RGB]{0,0,255}\bfseries,       % 关键字样式:蓝色加粗
  commentstyle=\color[RGB]{34,139,34}\itshape,      % 注释样式:绿色斜体
  stringstyle=\color[RGB]{178,34,34},               % 字符串样式:红色
  identifierstyle=\color{black},                    % 标识符样式:黑色
  %
  captionpos=b,                                      % 标题位置:下方
  xleftmargin=15pt,                                  % 左边距
  xrightmargin=5pt,                                  % 右边距
  keepspaces=true,                                   % 保留空格
  % escapeinside={@}{@},                               % 在代码中可使用 @…@ 转义 LaTeX
  aboveskip=1em,                                     % 代码块上方间距
  belowskip=1em                                      % 代码块下方间距(注意最后一行不要逗号)
}


% ======================
% 语言定义
% ======================

% --- Python ---
\lstdefinelanguage{Python}{
  keywords={False,None,True,and,as,assert,async,await,break,class,continue,def,
            del,elif,else,except,finally,for,from,global,if,import,in,is,lambda,
            nonlocal,not,or,pass,raise,return,try,while,with,yield},
  keywords=[2]{self,cls,__init__,__name__,__main__},
  keywords=[3]{print,input,len,str,int,float,list,dict,set,tuple,range},
  sensitive=true,
  comment=[l]{\#},
  morecomment=[s]{"""}{"""},
  morecomment=[s]{'''}{'''},
  string=[b]',
  string=[b]",
  morestring=[s]{r'}{'},
  morestring=[s]{r"}{"}
}

% --- MATLAB ---
\lstdefinelanguage{Matlab}{
  keywords={break,case,catch,classdef,continue,else,elseif,end,for,function,
            global,if,otherwise,parfor,persistent,return,spmd,switch,try,while},
  keywords=[2]{pi,eps,inf,nan,ans},
  keywords=[3]{plot,sin,cos,exp,log,mean,std},
  sensitive=false,
  comment=[l]{\%},
  morecomment=[s]{\%\{}{\%\}},
  string=[b]',
  string=[b]"
}

% --- C ---
\lstdefinelanguage{C}{
  keywords={auto,break,case,char,const,continue,default,do,double,else,enum,
            extern,float,for,goto,if,inline,int,long,register,restrict,return,
            short,signed,sizeof,static,struct,switch,typedef,union,unsigned,
            void,volatile,while,_Bool,_Complex,_Imaginary},
  sensitive=true,
  morecomment=[l]{//},
  morecomment=[s]{/*}{*/},
  string=[b]',
  string=[b]",
  alsoletter=\#,
  morekeywords=[2]{printf,scanf,malloc,free,sizeof}
}

% --- Mathematica ---
\lstdefinelanguage{Mathematica}{
  keywords={If,Do,For,While,Module,Block,With,Catch,Throw,Return,Break,Continue,
            Function,Pattern,Set,SetDelayed,Rule,RuleDelayed,ReplaceAll,
            ReplaceRepeated,True,False,Null,List,Table,Map,Apply,Flatten},
  morekeywords=[2]{Sin,Cos,Log,Integrate,D,Plot,Solve,Expand},
  sensitive=true,
  comment=[l]{(*)},
  morecomment=[l]{*)},
  string=[b]",
  alsoletter={@,?,\#},
  morestring=[b]`
}

% --- R ---
\lstdefinelanguage{R}{
  keywords={function,if,in,else,for,while,repeat,break,next,return,switch,
            try,tryCatch,stop,warning,require,library,attach,detach,source},
  keywords=[2]{c,matrix,data.frame,list,vector,array,table},
  keywords=[3]{mean,sd,plot,hist,lm,glm},
  sensitive=true,
  comment=[l]{\#},
  string=[b]',
  string=[b]",
  otherkeywords={!,!=,~,$,*,\&,\%/\%,\%*\%,\%\%,<-,<<-,->,->>,=,==,>,<,>=,<=}
}

% ======================
% 可分割算法环境
% ======================
\makeatletter
\newenvironment{breakablealgorithm}
  {% \begin{breakablealgorithm}
    \begin{center}
      \refstepcounter{algorithm}% New algorithm
      \hrule height.8pt depth0pt \kern4pt% \@fs@pre for \@fs@ruled
      \parskip 0pt
      \renewcommand{\caption}[2][\relax]{% Make a new \caption
        {\raggedright\textbf{\fname@algorithm~\thealgorithm: ##2}\par}%
        \ifx\relax##1\relax % #1 is \relax
          \addcontentsline{loa}{algorithm}{\protect\numberline{\thealgorithm}##2}%
        \else % #1 is not \relax
          \addcontentsline{loa}{algorithm}{\protect\numberline{\thealgorithm}##1}%
        \fi
        \kern2pt\hrule\kern2pt
     }
  }
  {% \end{breakablealgorithm}
     \kern2pt\hrule\relax% \@fs@post for \@fs@ruled
   \end{center}
  }
\makeatother

% 算法列表
\providecommand{\listofalgorithms}{\listof{algorithm}{List of Algorithms}}

% ======================
% 自定义代码环境
% ======================
\lstnewenvironment{pythoncode}[1][]    {\lstset{language=Python,#1}}{}
\lstnewenvironment{matlabcode}[1][]    {\lstset{language=Matlab,#1}}{}
\lstnewenvironment{ccode}[1][]         {\lstset{language=C,#1}}{}
\lstnewenvironment{mathematicacode}[1][]{\lstset{language=Mathematica,#1}}{}
\lstnewenvironment{rcode}[1][]         {\lstset{language=R,#1}}{}

% ======================
% 辅助命令
% ======================
\newcommand{\inlinecode}[1]{\colorbox{codebg}{\ttfamily #1}}
\newcommand{\inputcode}[2][]{\lstinputlisting[language=#1]{#2}}
                    % 代码环境
% % ======================
% 图表与浮动体设置
% ======================

% ---------- 标题样式 ----------
% 依赖 package.tex 中的 \usepackage{caption}
\captionsetup[figure]{
  font={small},          % 标题字体
  labelfont={bf},        % 标签加粗
  justification=centering % 居中对齐
}
\captionsetup[table]{
  font={small},
  labelfont={bf},
  justification=centering
}

% ---------- 自定义浮动体 ----------
% 依赖 package.tex 中的 \usepackage{float}
\newfloat{code}{htbp}{loc} % 定义 "code" 浮动体
\floatname{code}{程序}     % 浮动体标题名

% ---------- 自定义图片环境 ----------
\newenvironment{myfigure}[1][]{
  \begin{figure}[#1]
  \centering
}{
  \end{figure}
}
                  % 图片环境
% ======================
%  自定义样式与环境
% ======================

% ---------- 颜色定义 ----------
\usepackage{xcolor}
\definecolor{highlightcolor}{RGB}{255,230,179} % 淡黄高亮
\definecolor{myred}{RGB}{204,26,26}           % 强调文字

% ---------- 文字样式 ----------
\usepackage[normalem]{ulem}          % 波浪线 \uwave
\newcommand{\hl}[1]{\colorbox{highlightcolor}{#1}}
% \newcommand{\uline}[1]{\underline{#1}}
\newcommand{\wline}[1]{\uwave{#1}}
\newcommand{\myemph}[1]{\textcolor{myred}{#1}}

% ---------- 超链接 ----------
\usepackage{hyperref}
\hypersetup{
    colorlinks   = true,
    linkcolor    = black,
    citecolor    = black,
    urlcolor     = blue,
    filecolor    = magenta,
    pdfstartview = Fit,
    breaklinks   = true
}

% ---------- 行距 ----------
\usepackage{setspace}
\onehalfspacing                    % 1.5 倍行距
\setlength{\headheight}{14.5pt}   % fancyhdr 推荐值

% ---------- 图片搜索路径 ----------
\usepackage{graphicx}
\graphicspath{%
  {config/cover/}%
  {Chapters/chapter1/images/}%
  {Chapters/chapter2/images/}%
  {Chapters/chapter3/images/}%
  {images/}%
}
                  % 自定义文字环境

% -------------------- 封面模板选择 --------------------
% 默认使用索引为 1 的封面配置
\def\myIndex{1}
\input{\path/cover/cover_package_\myIndex.tex}

% ====================== End of _config.tex ======================


    % 单独编译时设置临时标题信息
    \title{第一章:XXXX}
    \author{\myAuthor}
    \date{\today}

    \begin{document}
    \maketitle
\fi

% ======= 正文章节 =======
\chapter{chapter}                                     %设置一章 
XXXXXXXX


%%%%%%%%%%%%%%%%%%%%%%%%%%%%%%%%%%%%%%%%%%%%%%%%%%%%%%%%%%%%%%%%%%%%%%%%%%%%%%%%%%%%%%%%%%%%%%%%%%%%%%%%%%%%%%%%%%%%%%%%

% ======= 单独编译模式结束 =======
\ifx\allfiles\undefined
    \end{document}
\fi
% ====================== End of chap1.tex ======================

% ====================== chapter2/chap2.tex ======================
% 第二章:基本概率分布模型
% 支持单独编译与整书编译两种模式
% ---------------------------------------------------------------

\ifx\allfiles\undefined
    % ======= 单独编译模式 =======
    \documentclass{ctexbook}

    % 载入全局配置(路径需根据文件结构调整)
    % ====================== _config.tex ======================
% 文档配置与全局参数定义
% ----------------------------------------------------------

% -------------------- 文档元信息 --------------------
\def\myTitle{主标题}               % 文档标题
\def\myAuthor{author}                                   % 作者
\def\myDateCover{封面日期:\today}                     % 封面日期
\def\myDateForeword{前言页显示日期:\today}             % 前言日期
\def\mySubheading{——副标题}               % 副标题

% -------------------- 前言内容 --------------------
\def\myForeword{前言标题}                              % 前言标题
\def\myForewordText{                                   % 前言正文
    这里是前言内容。
}

% -------------------- 导入全局宏包 --------------------
% package.tex

% =============================
% 数学相关宏包
% =============================
\usepackage{amsmath}          % 提供各种数学环境和命令
\usepackage{amssymb}          % 数学符号
\usepackage{amsthm}           % 定理、引理等环境
\usepackage{bm}               % 数学粗体
\usepackage{mathrsfs}         % 花体字母

% =============================
% 页面布局与格式
% =============================
\usepackage{geometry}         % 页面布局
\usepackage{setspace}         % 行距控制
\usepackage{graphicx}         % 插入图片
\usepackage{caption}          % 图表标题格式
\usepackage{subcaption}       % 子图标题
\usepackage{placeins}         % 控制浮动体位置
\usepackage{booktabs}         % 高质量表格线
\usepackage{enumitem}         % 自定义列表样式
\usepackage[bottom]{footmisc} % 脚注靠底
\usepackage{fancyhdr}         % 页眉页脚

% =============================
% 颜色与样式
% =============================
\usepackage[dvipsnames, svgnames]{xcolor} % 定义颜色
\usepackage{pifont}           % 特殊符号
\usepackage{tcolorbox}        % 彩色框
\tcbuselibrary{skins,breakable}

% =============================
% 代码与算法
% =============================
\usepackage{listings}         % 代码高亮
\usepackage{algorithm}        % 算法环境
\usepackage[noend]{algpseudocode} % 伪代码,无 end 标记
\usepackage{etoolbox}         % 宏编程工具

% =============================
% 其他功能
% =============================
\usepackage{hyperref}         % 超链接与交叉引用
\usepackage{url}              % 格式化 URL 和 DOI
\usepackage{float}            % 浮动体控制
\usepackage{appendix}         % 附录
\usepackage{animate}          % 动画支持


% =============================
% 可选功能(按需启用)
% =============================
\usepackage{tikz}           % 绘图
% \usepackage[most]{tcolorbox} % 更多 tcolorbox 功能
                              % 宏包加载

% -------------------- 导入环境配置 --------------------
% ======================
% 定理与盒子环境设置
% 依赖宏包:
%   amsthm, amssymb       (定理、证明环境)
%   xcolor                (颜色定义)
%   tcolorbox[skins,breakable] (彩色盒子)
% ======================

% ---------- 自定义颜色 ----------
\definecolor{greenshade}{rgb}{0.9, 1, 0.9}
\definecolor{green}{rgb}{0.2, 0.6, 0.2}
\definecolor{axmBack}{rgb}{0.95, 0.95, 1}
\definecolor{axmFrame}{rgb}{0.5, 0.5, 1}
\definecolor{axmTitle}{rgb}{0.2, 0.2, 0.6}
\definecolor{lightblue}{rgb}{0.8, 0.9, 1}
\definecolor{lightgray}{rgb}{0.95, 0.95, 0.95}
\definecolor{lightyellow}{rgb}{1, 0.98, 0.9}
\definecolor{lightcyan}{rgb}{0.9, 1, 1}
\definecolor{SkyBlue}{rgb}{0.2,0.6,1}

% ---------- amsthm 样式 ----------
% 斜体样式(定理、引理、命题、推论)
\newtheoremstyle{mythm}
  {\topsep}{\topsep}{\itshape}{}{\bfseries}{.}{.5em}{}
\theoremstyle{mythm}
\newtheorem{theorem}{定理}[chapter]
\newtheorem{lemma}[theorem]{引理}
\newtheorem{corollary}[theorem]{推论}
\newtheorem{proposition}[theorem]{命题}
\newtheorem{axiom}{公理}[chapter]        % 新增 axiom 环境

% 非斜体样式(定义、例、注)
\newtheoremstyle{mydef}
  {\topsep}{\topsep}{\upshape}{}{\bfseries}{.}{.5em}{}
\theoremstyle{mydef}
\newtheorem{definition}{定义}[chapter]
\newtheorem{example}{例}[chapter]
\newtheorem{remark}{注}[chapter]

% ---------- 自定义证明与解答环境 ----------
\renewenvironment{proof}{\indent\textcolor{SkyBlue}{\textbf{证明.}}\;}{\qed\par}
\newenvironment{solution}{\indent\textcolor{SkyBlue}{\textbf{解.}}\;}{\qed\par}

% ---------- tcolorbox 自定义环境 ----------
% 定义彩色盒子环境
\newtcolorbox[auto counter, number within=chapter, list inside=mydefbox]{defn}[1][]{
  enhanced, breakable,
  colback=greenshade, colframe=green, arc=3mm,
  title={\textbf{定义\thetcbcounter\ #1}},
  coltitle=black, fonttitle=\bfseries
}

% axiom 盒子
\newtcolorbox[auto counter, number within=chapter]{axiombox}[1][]{
  enhanced, breakable,
  colback=axmBack, colframe=axmFrame, arc=3mm,
  title={\textbf{公理\thetcbcounter\ #1}},
  coltitle=axmTitle, fonttitle=\bfseries,
}

% 注意框
\newtcolorbox{Notice}[1][注意]{
  colback=lightblue, colframe=black, arc=3mm,
  title={\textbf{#1}}, coltitle=white, fonttitle=\bfseries
}

% 证明框
\newtcolorbox{Proof}{
  colback=lightgray, colframe=black, arc=3mm,
  title={\textbf{证明}}, coltitle=black, fonttitle=\bfseries
}

% 推导框
\newtcolorbox{Derivation}[1][]{
  enhanced, breakable,
  colback=lightyellow, colframe=black, arc=3mm,
  title={\textbf{推导}}, coltitle=white, fonttitle=\bfseries,
  boxsep=2mm, #1
}

% 示例框
\newtcolorbox{examplebox}[1][示例]{
  colback=lightcyan, colframe=black, arc=3mm,
  title={\textbf{#1}}, coltitle=black, fonttitle=\bfseries
}

% ---------- 快捷别名 ----------
% \let\defn\definition
% \newenvironment{axiom}[1][]{\begin{axiombox}[#1]}{\end{axiombox}}
                 % 定理环境
% ======================
% 全局代码样式(基于 listings)
% ======================

\definecolor{codebg}{RGB}{245,245,245}       % 代码背景色
\definecolor{codeframe}{RGB}{220,220,220}    % 代码边框色

\lstset{
  basicstyle=\small\ttfamily,                         % 基本字体样式:小号等宽字体
  backgroundcolor=\color{codebg},                    % 代码背景色
  frame=single,                                      % 代码框架类型:单线框
  framesep=5pt,                                      % 代码框与框线的间距
  rulecolor=\color{codeframe},                       % 框线颜色
  framerule=0.8pt,                                   % 框线宽度
  %
  numbers=left,                                      % 行号显示在左边
  numberstyle=\footnotesize\color{gray},                     % 行号样式:灰色小号字体
  numbersep=5pt,                                     % 行号与代码的间距
  stepnumber=1,                                      % 每隔几行显示一个行号
  firstnumber=1,                                     % 起始行号
  %
  tabsize=4,                                         % Tab 等于几个空格
  showspaces=false,                                  % 不显示空格符号
  showstringspaces=false,                            % 字符串中的空格不显示特殊符号
  showtabs=false,                                    % 不显示 Tab 符号
  breaklines=true,                                   % 自动换行
  breakatwhitespace=false,                           % 是否仅在空格处换行
  prebreak=\raisebox{0ex}[0ex][0ex]{\ensuremath{\hookleftarrow}}, % 换行前标记符
  %
  keywordstyle=\color[RGB]{0,0,255}\bfseries,       % 关键字样式:蓝色加粗
  commentstyle=\color[RGB]{34,139,34}\itshape,      % 注释样式:绿色斜体
  stringstyle=\color[RGB]{178,34,34},               % 字符串样式:红色
  identifierstyle=\color{black},                    % 标识符样式:黑色
  %
  captionpos=b,                                      % 标题位置:下方
  xleftmargin=15pt,                                  % 左边距
  xrightmargin=5pt,                                  % 右边距
  keepspaces=true,                                   % 保留空格
  % escapeinside={@}{@},                               % 在代码中可使用 @…@ 转义 LaTeX
  aboveskip=1em,                                     % 代码块上方间距
  belowskip=1em                                      % 代码块下方间距(注意最后一行不要逗号)
}


% ======================
% 语言定义
% ======================

% --- Python ---
\lstdefinelanguage{Python}{
  keywords={False,None,True,and,as,assert,async,await,break,class,continue,def,
            del,elif,else,except,finally,for,from,global,if,import,in,is,lambda,
            nonlocal,not,or,pass,raise,return,try,while,with,yield},
  keywords=[2]{self,cls,__init__,__name__,__main__},
  keywords=[3]{print,input,len,str,int,float,list,dict,set,tuple,range},
  sensitive=true,
  comment=[l]{\#},
  morecomment=[s]{"""}{"""},
  morecomment=[s]{'''}{'''},
  string=[b]',
  string=[b]",
  morestring=[s]{r'}{'},
  morestring=[s]{r"}{"}
}

% --- MATLAB ---
\lstdefinelanguage{Matlab}{
  keywords={break,case,catch,classdef,continue,else,elseif,end,for,function,
            global,if,otherwise,parfor,persistent,return,spmd,switch,try,while},
  keywords=[2]{pi,eps,inf,nan,ans},
  keywords=[3]{plot,sin,cos,exp,log,mean,std},
  sensitive=false,
  comment=[l]{\%},
  morecomment=[s]{\%\{}{\%\}},
  string=[b]',
  string=[b]"
}

% --- C ---
\lstdefinelanguage{C}{
  keywords={auto,break,case,char,const,continue,default,do,double,else,enum,
            extern,float,for,goto,if,inline,int,long,register,restrict,return,
            short,signed,sizeof,static,struct,switch,typedef,union,unsigned,
            void,volatile,while,_Bool,_Complex,_Imaginary},
  sensitive=true,
  morecomment=[l]{//},
  morecomment=[s]{/*}{*/},
  string=[b]',
  string=[b]",
  alsoletter=\#,
  morekeywords=[2]{printf,scanf,malloc,free,sizeof}
}

% --- Mathematica ---
\lstdefinelanguage{Mathematica}{
  keywords={If,Do,For,While,Module,Block,With,Catch,Throw,Return,Break,Continue,
            Function,Pattern,Set,SetDelayed,Rule,RuleDelayed,ReplaceAll,
            ReplaceRepeated,True,False,Null,List,Table,Map,Apply,Flatten},
  morekeywords=[2]{Sin,Cos,Log,Integrate,D,Plot,Solve,Expand},
  sensitive=true,
  comment=[l]{(*)},
  morecomment=[l]{*)},
  string=[b]",
  alsoletter={@,?,\#},
  morestring=[b]`
}

% --- R ---
\lstdefinelanguage{R}{
  keywords={function,if,in,else,for,while,repeat,break,next,return,switch,
            try,tryCatch,stop,warning,require,library,attach,detach,source},
  keywords=[2]{c,matrix,data.frame,list,vector,array,table},
  keywords=[3]{mean,sd,plot,hist,lm,glm},
  sensitive=true,
  comment=[l]{\#},
  string=[b]',
  string=[b]",
  otherkeywords={!,!=,~,$,*,\&,\%/\%,\%*\%,\%\%,<-,<<-,->,->>,=,==,>,<,>=,<=}
}

% ======================
% 可分割算法环境
% ======================
\makeatletter
\newenvironment{breakablealgorithm}
  {% \begin{breakablealgorithm}
    \begin{center}
      \refstepcounter{algorithm}% New algorithm
      \hrule height.8pt depth0pt \kern4pt% \@fs@pre for \@fs@ruled
      \parskip 0pt
      \renewcommand{\caption}[2][\relax]{% Make a new \caption
        {\raggedright\textbf{\fname@algorithm~\thealgorithm: ##2}\par}%
        \ifx\relax##1\relax % #1 is \relax
          \addcontentsline{loa}{algorithm}{\protect\numberline{\thealgorithm}##2}%
        \else % #1 is not \relax
          \addcontentsline{loa}{algorithm}{\protect\numberline{\thealgorithm}##1}%
        \fi
        \kern2pt\hrule\kern2pt
     }
  }
  {% \end{breakablealgorithm}
     \kern2pt\hrule\relax% \@fs@post for \@fs@ruled
   \end{center}
  }
\makeatother

% 算法列表
\providecommand{\listofalgorithms}{\listof{algorithm}{List of Algorithms}}

% ======================
% 自定义代码环境
% ======================
\lstnewenvironment{pythoncode}[1][]    {\lstset{language=Python,#1}}{}
\lstnewenvironment{matlabcode}[1][]    {\lstset{language=Matlab,#1}}{}
\lstnewenvironment{ccode}[1][]         {\lstset{language=C,#1}}{}
\lstnewenvironment{mathematicacode}[1][]{\lstset{language=Mathematica,#1}}{}
\lstnewenvironment{rcode}[1][]         {\lstset{language=R,#1}}{}

% ======================
% 辅助命令
% ======================
\newcommand{\inlinecode}[1]{\colorbox{codebg}{\ttfamily #1}}
\newcommand{\inputcode}[2][]{\lstinputlisting[language=#1]{#2}}
                    % 代码环境
% % ======================
% 图表与浮动体设置
% ======================

% ---------- 标题样式 ----------
% 依赖 package.tex 中的 \usepackage{caption}
\captionsetup[figure]{
  font={small},          % 标题字体
  labelfont={bf},        % 标签加粗
  justification=centering % 居中对齐
}
\captionsetup[table]{
  font={small},
  labelfont={bf},
  justification=centering
}

% ---------- 自定义浮动体 ----------
% 依赖 package.tex 中的 \usepackage{float}
\newfloat{code}{htbp}{loc} % 定义 "code" 浮动体
\floatname{code}{程序}     % 浮动体标题名

% ---------- 自定义图片环境 ----------
\newenvironment{myfigure}[1][]{
  \begin{figure}[#1]
  \centering
}{
  \end{figure}
}
                  % 图片环境
% ======================
%  自定义样式与环境
% ======================

% ---------- 颜色定义 ----------
\usepackage{xcolor}
\definecolor{highlightcolor}{RGB}{255,230,179} % 淡黄高亮
\definecolor{myred}{RGB}{204,26,26}           % 强调文字

% ---------- 文字样式 ----------
\usepackage[normalem]{ulem}          % 波浪线 \uwave
\newcommand{\hl}[1]{\colorbox{highlightcolor}{#1}}
% \newcommand{\uline}[1]{\underline{#1}}
\newcommand{\wline}[1]{\uwave{#1}}
\newcommand{\myemph}[1]{\textcolor{myred}{#1}}

% ---------- 超链接 ----------
\usepackage{hyperref}
\hypersetup{
    colorlinks   = true,
    linkcolor    = black,
    citecolor    = black,
    urlcolor     = blue,
    filecolor    = magenta,
    pdfstartview = Fit,
    breaklinks   = true
}

% ---------- 行距 ----------
\usepackage{setspace}
\onehalfspacing                    % 1.5 倍行距
\setlength{\headheight}{14.5pt}   % fancyhdr 推荐值

% ---------- 图片搜索路径 ----------
\usepackage{graphicx}
\graphicspath{%
  {config/cover/}%
  {Chapters/chapter1/images/}%
  {Chapters/chapter2/images/}%
  {Chapters/chapter3/images/}%
  {images/}%
}
                  % 自定义文字环境

% -------------------- 封面模板选择 --------------------
% 默认使用索引为 1 的封面配置
\def\myIndex{1}
\input{\path/cover/cover_package_\myIndex.tex}

% ====================== End of _config.tex ======================


    % 单独编译时设置临时标题信息
    \title{第二章:XXXXXXXXX}
    \author{\myAuthor}
    \date{\today}

    \begin{document}
    \maketitle
\fi

% ======= 正文章节 =======
\chapter{chapter}


Main Text




%%%%%%%%%%%%%%%%%%%%%%%%%%%%%%%%%%%%%%%%%%%%%%%%%%%%%%%%%%%%%%%%%%%%%%%%%%%%%%%%%%%%%%%%%%%%%%%%%%%%%%%%%%%%%%%%%%%%%%%%

% ======= 单独编译模式结束 =======
\ifx\allfiles\undefined
    \end{document}
\fi
% ====================== End of chap2.tex ======================
\input{Chapters/chapter3/chap3.tex}
% ...

% ======================== 附录(可选) =====================
\appendix % 切换到附录模式
% ======================
% 全局代码样式(基于 listings)
% ======================

\definecolor{codebg}{RGB}{245,245,245}       % 代码背景色
\definecolor{codeframe}{RGB}{220,220,220}    % 代码边框色

\lstset{
  basicstyle=\small\ttfamily,                         % 基本字体样式:小号等宽字体
  backgroundcolor=\color{codebg},                    % 代码背景色
  frame=single,                                      % 代码框架类型:单线框
  framesep=5pt,                                      % 代码框与框线的间距
  rulecolor=\color{codeframe},                       % 框线颜色
  framerule=0.8pt,                                   % 框线宽度
  %
  numbers=left,                                      % 行号显示在左边
  numberstyle=\footnotesize\color{gray},                     % 行号样式:灰色小号字体
  numbersep=5pt,                                     % 行号与代码的间距
  stepnumber=1,                                      % 每隔几行显示一个行号
  firstnumber=1,                                     % 起始行号
  %
  tabsize=4,                                         % Tab 等于几个空格
  showspaces=false,                                  % 不显示空格符号
  showstringspaces=false,                            % 字符串中的空格不显示特殊符号
  showtabs=false,                                    % 不显示 Tab 符号
  breaklines=true,                                   % 自动换行
  breakatwhitespace=false,                           % 是否仅在空格处换行
  prebreak=\raisebox{0ex}[0ex][0ex]{\ensuremath{\hookleftarrow}}, % 换行前标记符
  %
  keywordstyle=\color[RGB]{0,0,255}\bfseries,       % 关键字样式:蓝色加粗
  commentstyle=\color[RGB]{34,139,34}\itshape,      % 注释样式:绿色斜体
  stringstyle=\color[RGB]{178,34,34},               % 字符串样式:红色
  identifierstyle=\color{black},                    % 标识符样式:黑色
  %
  captionpos=b,                                      % 标题位置:下方
  xleftmargin=15pt,                                  % 左边距
  xrightmargin=5pt,                                  % 右边距
  keepspaces=true,                                   % 保留空格
  % escapeinside={@}{@},                               % 在代码中可使用 @…@ 转义 LaTeX
  aboveskip=1em,                                     % 代码块上方间距
  belowskip=1em                                      % 代码块下方间距(注意最后一行不要逗号)
}


% ======================
% 语言定义
% ======================

% --- Python ---
\lstdefinelanguage{Python}{
  keywords={False,None,True,and,as,assert,async,await,break,class,continue,def,
            del,elif,else,except,finally,for,from,global,if,import,in,is,lambda,
            nonlocal,not,or,pass,raise,return,try,while,with,yield},
  keywords=[2]{self,cls,__init__,__name__,__main__},
  keywords=[3]{print,input,len,str,int,float,list,dict,set,tuple,range},
  sensitive=true,
  comment=[l]{\#},
  morecomment=[s]{"""}{"""},
  morecomment=[s]{'''}{'''},
  string=[b]',
  string=[b]",
  morestring=[s]{r'}{'},
  morestring=[s]{r"}{"}
}

% --- MATLAB ---
\lstdefinelanguage{Matlab}{
  keywords={break,case,catch,classdef,continue,else,elseif,end,for,function,
            global,if,otherwise,parfor,persistent,return,spmd,switch,try,while},
  keywords=[2]{pi,eps,inf,nan,ans},
  keywords=[3]{plot,sin,cos,exp,log,mean,std},
  sensitive=false,
  comment=[l]{\%},
  morecomment=[s]{\%\{}{\%\}},
  string=[b]',
  string=[b]"
}

% --- C ---
\lstdefinelanguage{C}{
  keywords={auto,break,case,char,const,continue,default,do,double,else,enum,
            extern,float,for,goto,if,inline,int,long,register,restrict,return,
            short,signed,sizeof,static,struct,switch,typedef,union,unsigned,
            void,volatile,while,_Bool,_Complex,_Imaginary},
  sensitive=true,
  morecomment=[l]{//},
  morecomment=[s]{/*}{*/},
  string=[b]',
  string=[b]",
  alsoletter=\#,
  morekeywords=[2]{printf,scanf,malloc,free,sizeof}
}

% --- Mathematica ---
\lstdefinelanguage{Mathematica}{
  keywords={If,Do,For,While,Module,Block,With,Catch,Throw,Return,Break,Continue,
            Function,Pattern,Set,SetDelayed,Rule,RuleDelayed,ReplaceAll,
            ReplaceRepeated,True,False,Null,List,Table,Map,Apply,Flatten},
  morekeywords=[2]{Sin,Cos,Log,Integrate,D,Plot,Solve,Expand},
  sensitive=true,
  comment=[l]{(*)},
  morecomment=[l]{*)},
  string=[b]",
  alsoletter={@,?,\#},
  morestring=[b]`
}

% --- R ---
\lstdefinelanguage{R}{
  keywords={function,if,in,else,for,while,repeat,break,next,return,switch,
            try,tryCatch,stop,warning,require,library,attach,detach,source},
  keywords=[2]{c,matrix,data.frame,list,vector,array,table},
  keywords=[3]{mean,sd,plot,hist,lm,glm},
  sensitive=true,
  comment=[l]{\#},
  string=[b]',
  string=[b]",
  otherkeywords={!,!=,~,$,*,\&,\%/\%,\%*\%,\%\%,<-,<<-,->,->>,=,==,>,<,>=,<=}
}

% ======================
% 可分割算法环境
% ======================
\makeatletter
\newenvironment{breakablealgorithm}
  {% \begin{breakablealgorithm}
    \begin{center}
      \refstepcounter{algorithm}% New algorithm
      \hrule height.8pt depth0pt \kern4pt% \@fs@pre for \@fs@ruled
      \parskip 0pt
      \renewcommand{\caption}[2][\relax]{% Make a new \caption
        {\raggedright\textbf{\fname@algorithm~\thealgorithm: ##2}\par}%
        \ifx\relax##1\relax % #1 is \relax
          \addcontentsline{loa}{algorithm}{\protect\numberline{\thealgorithm}##2}%
        \else % #1 is not \relax
          \addcontentsline{loa}{algorithm}{\protect\numberline{\thealgorithm}##1}%
        \fi
        \kern2pt\hrule\kern2pt
     }
  }
  {% \end{breakablealgorithm}
     \kern2pt\hrule\relax% \@fs@post for \@fs@ruled
   \end{center}
  }
\makeatother

% 算法列表
\providecommand{\listofalgorithms}{\listof{algorithm}{List of Algorithms}}

% ======================
% 自定义代码环境
% ======================
\lstnewenvironment{pythoncode}[1][]    {\lstset{language=Python,#1}}{}
\lstnewenvironment{matlabcode}[1][]    {\lstset{language=Matlab,#1}}{}
\lstnewenvironment{ccode}[1][]         {\lstset{language=C,#1}}{}
\lstnewenvironment{mathematicacode}[1][]{\lstset{language=Mathematica,#1}}{}
\lstnewenvironment{rcode}[1][]         {\lstset{language=R,#1}}{}

% ======================
% 辅助命令
% ======================
\newcommand{\inlinecode}[1]{\colorbox{codebg}{\ttfamily #1}}
\newcommand{\inputcode}[2][]{\lstinputlisting[language=#1]{#2}}
                 % 引入代码附录
% \input{Appendix/Data/data}                 % 引入数据附录
% \input{Appendix/Proofs/proofs}             % 引入证明附录
% 附录 D: 补充说明
\newpage
\section{附录 D: 补充说明}
\label{app:supplementary}

\subsection{supplementary title}
\label{sec:label of supplementary}

11111
% main text % 引入补充说明附录

% ======================== 参考文献(可选) ================
\cleardoublepage % 确保参考文献从新的一页开始
\phantomsection % 用于 hyperref 宏包正确设置参考文献的链接
\addcontentsline{toc}{chapter}{参考书} % 将参考书添加到目录中
\bibliographystyle{plain} % 选择参考文献的样式
\bibliography{References/Books/books.bib} % 引入参考书文件

% \cleardoublepage % 确保参考文献从新的一页开始
% \phantomsection
% \addcontentsline{toc}{chapter}{参考论文}
% \bibliography{References/Papers/papers.bib}

% \cleardoublepage
% \phantomsection
% \addcontentsline{toc}{chapter}{参考网站}
% \bibliography{References/Websites/websites}

\end{document}
% ====================== End of main.tex ====================
