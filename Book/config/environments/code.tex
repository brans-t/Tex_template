% ======================
% 全局代码样式(基于 listings)
% ======================

\definecolor{codebg}{RGB}{245,245,245}       % 代码背景色
\definecolor{codeframe}{RGB}{220,220,220}    % 代码边框色

\lstset{
  basicstyle=\small\ttfamily,                         % 基本字体样式:小号等宽字体
  backgroundcolor=\color{codebg},                    % 代码背景色
  frame=single,                                      % 代码框架类型:单线框
  framesep=5pt,                                      % 代码框与框线的间距
  rulecolor=\color{codeframe},                       % 框线颜色
  framerule=0.8pt,                                   % 框线宽度
  %
  numbers=left,                                      % 行号显示在左边
  numberstyle=\footnotesize\color{gray},                     % 行号样式:灰色小号字体
  numbersep=5pt,                                     % 行号与代码的间距
  stepnumber=1,                                      % 每隔几行显示一个行号
  firstnumber=1,                                     % 起始行号
  %
  tabsize=4,                                         % Tab 等于几个空格
  showspaces=false,                                  % 不显示空格符号
  showstringspaces=false,                            % 字符串中的空格不显示特殊符号
  showtabs=false,                                    % 不显示 Tab 符号
  breaklines=true,                                   % 自动换行
  breakatwhitespace=false,                           % 是否仅在空格处换行
  prebreak=\raisebox{0ex}[0ex][0ex]{\ensuremath{\hookleftarrow}}, % 换行前标记符
  %
  keywordstyle=\color[RGB]{0,0,255}\bfseries,       % 关键字样式:蓝色加粗
  commentstyle=\color[RGB]{34,139,34}\itshape,      % 注释样式:绿色斜体
  stringstyle=\color[RGB]{178,34,34},               % 字符串样式:红色
  identifierstyle=\color{black},                    % 标识符样式:黑色
  %
  captionpos=b,                                      % 标题位置:下方
  xleftmargin=15pt,                                  % 左边距
  xrightmargin=5pt,                                  % 右边距
  keepspaces=true,                                   % 保留空格
  % escapeinside={@}{@},                               % 在代码中可使用 @…@ 转义 LaTeX
  aboveskip=1em,                                     % 代码块上方间距
  belowskip=1em                                      % 代码块下方间距(注意最后一行不要逗号)
}


% ======================
% 语言定义
% ======================

% --- Python ---
\lstdefinelanguage{Python}{
  keywords={False,None,True,and,as,assert,async,await,break,class,continue,def,
            del,elif,else,except,finally,for,from,global,if,import,in,is,lambda,
            nonlocal,not,or,pass,raise,return,try,while,with,yield},
  keywords=[2]{self,cls,__init__,__name__,__main__},
  keywords=[3]{print,input,len,str,int,float,list,dict,set,tuple,range},
  sensitive=true,
  comment=[l]{\#},
  morecomment=[s]{"""}{"""},
  morecomment=[s]{'''}{'''},
  string=[b]',
  string=[b]",
  morestring=[s]{r'}{'},
  morestring=[s]{r"}{"}
}

% --- MATLAB ---
\lstdefinelanguage{Matlab}{
  keywords={break,case,catch,classdef,continue,else,elseif,end,for,function,
            global,if,otherwise,parfor,persistent,return,spmd,switch,try,while},
  keywords=[2]{pi,eps,inf,nan,ans},
  keywords=[3]{plot,sin,cos,exp,log,mean,std},
  sensitive=false,
  comment=[l]{\%},
  morecomment=[s]{\%\{}{\%\}},
  string=[b]',
  string=[b]"
}

% --- C ---
\lstdefinelanguage{C}{
  keywords={auto,break,case,char,const,continue,default,do,double,else,enum,
            extern,float,for,goto,if,inline,int,long,register,restrict,return,
            short,signed,sizeof,static,struct,switch,typedef,union,unsigned,
            void,volatile,while,_Bool,_Complex,_Imaginary},
  sensitive=true,
  morecomment=[l]{//},
  morecomment=[s]{/*}{*/},
  string=[b]',
  string=[b]",
  alsoletter=\#,
  morekeywords=[2]{printf,scanf,malloc,free,sizeof}
}

% --- Mathematica ---
\lstdefinelanguage{Mathematica}{
  keywords={If,Do,For,While,Module,Block,With,Catch,Throw,Return,Break,Continue,
            Function,Pattern,Set,SetDelayed,Rule,RuleDelayed,ReplaceAll,
            ReplaceRepeated,True,False,Null,List,Table,Map,Apply,Flatten},
  morekeywords=[2]{Sin,Cos,Log,Integrate,D,Plot,Solve,Expand},
  sensitive=true,
  comment=[l]{(*)},
  morecomment=[l]{*)},
  string=[b]",
  alsoletter={@,?,\#},
  morestring=[b]`
}

% --- R ---
\lstdefinelanguage{R}{
  keywords={function,if,in,else,for,while,repeat,break,next,return,switch,
            try,tryCatch,stop,warning,require,library,attach,detach,source},
  keywords=[2]{c,matrix,data.frame,list,vector,array,table},
  keywords=[3]{mean,sd,plot,hist,lm,glm},
  sensitive=true,
  comment=[l]{\#},
  string=[b]',
  string=[b]",
  otherkeywords={!,!=,~,$,*,\&,\%/\%,\%*\%,\%\%,<-,<<-,->,->>,=,==,>,<,>=,<=}
}

% ======================
% 可分割算法环境
% ======================
\makeatletter
\newenvironment{breakablealgorithm}
  {% \begin{breakablealgorithm}
    \begin{center}
      \refstepcounter{algorithm}% New algorithm
      \hrule height.8pt depth0pt \kern4pt% \@fs@pre for \@fs@ruled
      \parskip 0pt
      \renewcommand{\caption}[2][\relax]{% Make a new \caption
        {\raggedright\textbf{\fname@algorithm~\thealgorithm: ##2}\par}%
        \ifx\relax##1\relax % #1 is \relax
          \addcontentsline{loa}{algorithm}{\protect\numberline{\thealgorithm}##2}%
        \else % #1 is not \relax
          \addcontentsline{loa}{algorithm}{\protect\numberline{\thealgorithm}##1}%
        \fi
        \kern2pt\hrule\kern2pt
     }
  }
  {% \end{breakablealgorithm}
     \kern2pt\hrule\relax% \@fs@post for \@fs@ruled
   \end{center}
  }
\makeatother

% 算法列表
\providecommand{\listofalgorithms}{\listof{algorithm}{List of Algorithms}}

% ======================
% 自定义代码环境
% ======================
\lstnewenvironment{pythoncode}[1][]    {\lstset{language=Python,#1}}{}
\lstnewenvironment{matlabcode}[1][]    {\lstset{language=Matlab,#1}}{}
\lstnewenvironment{ccode}[1][]         {\lstset{language=C,#1}}{}
\lstnewenvironment{mathematicacode}[1][]{\lstset{language=Mathematica,#1}}{}
\lstnewenvironment{rcode}[1][]         {\lstset{language=R,#1}}{}

% ======================
% 辅助命令
% ======================
\newcommand{\inlinecode}[1]{\colorbox{codebg}{\ttfamily #1}}
\newcommand{\inputcode}[2][]{\lstinputlisting[language=#1]{#2}}
